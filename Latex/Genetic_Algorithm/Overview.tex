\section{Genetic Algorithms}

Genetic Algorithms work by evolving a pool of candidate solutions, optimising them in the process till it eventually converges to an optimal enough solution. Optimising \textit{usually} means minimising, unless otherwise specified, and this is the assumption throughout these notes. To convert from a maximisation to a minimisation problem, just multiply the cost function by -1 .
Each solution is represented as a \textbf{chromosome}, which contains values for the variables of the given problem (each variable is known as a \textbf{gene}). For example, if we're optimising a function of the form \( f(x,y) \), then each chromosome contains two genes. \textbf{Binary Genetic Algorithms} represent chromosomes as a binary string, \textbf{Continuous Genetic Algorithms} represent them as a tuple (or list or array) of genes.
\\ \\
The general form of a genetic algorithm has 4 main parts:
\begin{enumerate}
    \item Natural Selection
    \item Pair Selection
    \item Crossover
    \item Mutation\\ 
\end{enumerate}
Combining all of these together, the overall flow of the algorithm is: 
\begin{enumerate}[label=\Alph*]
\item Create a random initial population
\item While stopping criteria are unmet:
\begin{enumerate}[label=\arabic*]
    \item Filter population with Natural Selection
    \item Create pairs via Pair Selection
    \item Make new chromosomes via Crossover
    \item Apply mutation to the population
\end{enumerate}
\item Select the best performing chromosome as the solution\\
\end{enumerate}
For problems that have constraints or multiple objectives to be optimised, having the cost function be a weighted sum of each individual requirement is sufficient. Note that all must be maximisation/minimisation: these can be converted with negative weights if required. Constraints should degrade the cost if not met (e.g add a large integer with absolute value higher than the lowest possible cost), and should be designed such that a solution violating a constraint cannot have a lower cost than a function that satisfies it. There is no such restriction for preferences. 

\subsection{Stopping Criteria}
Any/all of the below can be used: it depends on the situation:
\begin{enumerate}
    \item An acceptable (low enough cost) solution has been found
    \item The mean/standard deviation/range of the chromosomes is within a target value
    \item No change in chromosomes/best/cost/worst cost after X iterations
    \item N iterations have elapsed
\end{enumerate}