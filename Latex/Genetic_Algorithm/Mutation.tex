\subsection{Mutation}
Mutation is the process that actually drives evolution: by changing the chromosomes into new values, new solutions can be evaluated and used. It is possible to have an algorithm that consists entirely of mutation: this would perform very similar to random walk or (1+1)ES. Crossovers move the genes into potentially better solutions, mutations discover new solutions. 

\subsubsection{How many Mutations?}
Mutation is determined using a mutation rate, $\mu$. To prevent good solutions from being mutated, we can designate the top n chromosomes as \emph{elite}, and not consider them for mutation. n can be left as 0, in which case any chromosome can be mutated. 

With $N_{pop}$ as the size of the population, the standard formula is:

\begin{equation}
    M = (N_{pop} - n) * \mu * N_{bits}
\end{equation}

$N_{bits}$ is the length of the binary string for a chromosome for BGA, and the number of genes in a chromosome for CGA.

\subsubsection{Binary Mutation}
Disregarding the top n chromosomes, choose M bits randomly across the population and flip them (0$\rightarrow$1, 1$\rightarrow$0). 

\subsubsection{Numeric Mutation}
This uses a variable $\sigma$ to control how much a gene mutates. Disregarding the top n chromosomes, choose M genes randomly and apply:

\[
    Gene = Gene + \sigma * N_n(0,1)
\]

$\sigma$ is usually a constant, but for a dynamic system can be changed between generations. $N_n$ is a random number generator, but generates numbers according to a standard distribution (i.e 0.5 is most likely to be generated, 0 and 1 are very unlikely). 