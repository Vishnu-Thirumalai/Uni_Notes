\subsection{Genetic Programming}
This application of GA allows it to build a program for us. It is provided an input and a desired output, and the algorithm builds us a program to convert between to two.

Rather than strings or tuples, chromosomes are represented as tress of varying sizes and shapes. This requires extra overhead for each chromosome, since the size and shape has to be tracked as well.
Lead nodes are variables and constants, while internal tree nodes are functions that combine these.

The algorithm also has to be provided a set of possible values for variables/constants, a set of functions, and semantics on how to combine the two. e.g. the SUM function requires two numerics and outputs a numeric, the OR function requires two binary strings and outputs a binary string, etc.

\subsubsection{Crossovers}
The simplest crossover is to swap and replace subtrees between chromosomes. A single child can be generated by replacing a subtree in one parent by one from the other parent, and two children can be obtained by swapping subtrees.

\subsubsection{Mutation}
\begin{description}
    \item [Node Mutation] Swap a node with an appropriate alternative\\ e.g replace SUM with DIFF, replace 4 with 11
    \item[Swap Mutation] Swap the order of terminal nodes provided to a function\\ e.g a DIFF b $\rightarrow$ b DIFF a
    \item[Grow Mutation] Replace a terminal node with a new subtree that provides the same type as the replaced node e.g. replace 5 with a SUM 7
    \item[Gaussian Mutation] Add a value determined by a gaussian ($N_n * \sigma)$ to a terminal node
    \item[Trunc Mutation] Replace a subtree with an appropriate terminal node\\e.g replace 5 MOD 2 with 11
\end{description}