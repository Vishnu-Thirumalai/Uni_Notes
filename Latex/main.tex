\documentclass[12pt, a4paper, titlepage]{article}
\usepackage[utf8]{inputenc}

\usepackage{amsmath}
\usepackage{float}
\usepackage{datetime}
\usepackage{enumitem}
\usepackage{varwidth}

\usepackage{geometry}
\geometry{top=1in, left = 1in, right = 1in}




\title{Biologically Inspired Methods: Cheatsheet}
\date{\today}
\author{Vishnu}


\begin{document}

\maketitle

\section{Exploitation vs. Exploration}
All guided searching algorithms try to strike a balance between these.
\emph{Exploitation} is taking known factors and using them to intensify them search: for example, on finding a good potential solution, search the area around it for better solutions. This allows us to follow trends in the search space, but often leads to being stuck in local minima.

\emph{Exploration} involves diversifying the search by visiting previously unknown areas and trying different values to discover potentially better solutions. Doing this too much leads to an effectively random search, but in moderation this allows us to identify and break out of local minima.

\section{Flow Networks}
Flow networks are directed graphs with two additions:
\begin{itemize}
    \item Each edge (u,v) has a capacity c(u,v)
    \item Two distinct nodes are marked as the source and sink 
\end{itemize}
Usually, optimising a flow network is finding the maximum flow: the amount of units that can travel from the source to the sink simultaneously. Units travel along the edges, but an edge can only support up to its capacity simultaneously. Other problems may be the minimum flow that hits a certain target, or having multiple types of flow at the same time. \\
Most of the below algorithms assume that if an edge (a,b) exists, then the edge (b,a) also exists. If it's not present in the given graph, it's given a default capacity of zero.

\subsection{Flow}
The flow f(u,v) of an edge (u,v) is the amount of units passing through the edge: it's abstracted to a function that maps edges to real numbers. Flow is subject to the following:
\begin{enumerate}
    \item f(a,b) $\leq$ c(a,b) $\forall$ (a,b) - Flow for an edge is always less than/equal to capacity
    \item For all nodes except source/sink, incoming flow = outgoing flow (net flow on a node = 0)
    \item If f(a,b) $>$ 0, then f(b,a) = 0
\end{enumerate}
A \textbf{saturated edge} is one where f(a,b) = c(a,b). The \textbf{value of a flow} in a network is defined as the net flow from the source/net flow into the sink.\\
A flow can be decomposed into \textbf{paths or cycles}, which start from the source and end at the sink. There can be at most m paths/cycles, where m is the number of edges.





\subsection{Binary Strings}
For the \textbf{Binary Genetic Algorithm}, we also have to determine how to convert to and from binary strings. Using binary strings is mostly historical, back when the memory used by each chromosome was important. It's also fairly simple to implement.

\subsubsection{Binary String Length}
The overall length of the binary string is dependent on the number of possible values our gene can take. For example: for an equation \(f(x,y)\), if x has 100 possible values and y has 30, the overall length of the binary string representation will be \(7+5=12\). 
\\
Normally though, we're given a range and precision that our genes must adhere to. Re-phrasing our earlier example: \(0 \leq x \leq 1\) with a precision of 2 decimal places, \(5 \leq y \leq 8\) with a precision of 1 decimal place. To convert from these to the above, we use the following equation:

\begin{equation}
    Number\_of\_Values = \frac{x_{hi} - x_{lo}}{10^{-d}} 
\end{equation}

This equation is the same as saying \( integer\_range * 10^{decimal\_places}\). Now that we have the number of possible values a gene can take, we need to find the minimum length of a binary string that can represent that many distinct values. A binary string of length \textbf{n} can represent \(2^n-1\) distinct values. 

For example: using the above equation, we see there are 30 distinct values for y. \(2^4-1\) is 15, \(2^5-1\) is 31. Therefore we choose a binary string of length 5 to represent y. We could have also calculated this as \( ceil(log_2(30))\).

The above can be also be summarised as: 

\begin{equation}
    \frac{x_{hi} - x_{lo}}{10^{-d}}  \le 2^m - 1 
\end{equation}

Where m is the required length. To get the full string, do this for each variable and add them together.

\subsubsection{Binary Encoding and Decoding}
Our binary representation roughly contains one unique string for each possible value. It's not going to be a perfect match though, as binary can't perfectly split decimal. To ease in the conversion, we calculate the \textbf{Binary Value}:

\[
   \frac{x_{hi} - x_{lo}}{2^m - 1}
\]

This gives us the decimal value of a 1 in the binary string, and we'll represent it as $b$. Representing the binary string as $BS$ and the decimal value as $D$, the conversions are:

\begin{align}
    Encoding&:  BS = binary(round(\frac{D -  x_{lo}}{b}))\\
    Decoding&:  D  = x_{lo} + decimal(BS) * b
\end{align}

With these, $x_{lo}$ is represented as 0\dots0 and $x_{hi}$ as 1\dots1 for every gene. The complete binary string for a chromosome is just the concatenation of the strings for each gene.

\subsubsection{Grey Codes}
One problem with binary strings is adjacent numbers can have very different representations. For example: 3 and 4 have a difference of 1 in decimal, but their binary representations have every bit different (011 and 100). We can quantify this as the \textbf{Hamming Distance}, which is the number of bits that are different between two binary numbers (e.g 011 and 100 have a h.d of 3, 001 and 101 have a h.d of 1). 

Grey codes are an alternate representation that guarantee consecutive numbers to have a Hamming Distance of 1. Representing the binary string as $b$ and the grey code as $g$, both with length $n$, the conversions are as follows:

Binary to Grey:
\begin{equation}
    g[i] = 
      \begin{cases}
      b[i], & \text{if}\ i = n \\
      b[i]\, XOR\, g[i+1], & \text{otherwise}
    \end{cases}
\end{equation}

Grey to Binary:
\begin{equation}
    b[i] = 
      \begin{cases}
      g[i], & \text{if}\ i = n \\
      b[i+1]\, XOR\, g[i], & \text{otherwise}
    \end{cases}    
\end{equation}


\subsection{Natural Selection}
This part of the algorithm determines which chromosomes are carried forward to the next generation.

\subsubsection{$X_{rate}$}
Only the best survive. The top (lowest cost) $N_{keep}$ chromosomes are chosen, and the rest discarded. This method guarantees a fixed amount from each generation are kept, though this can be varied by changing $X_{rate}$ (a user-defined percentage) at each iteration. With $N_{pop}$ as the size of the population, $N_{keep}$ can be calculated as:

\[
    N_{keep} = N_{pop} * X_{rate}
\]

\subsection{Thresholding}
Only chromosomes with a score below a given threshold survive. This has the advantage of not having to sort and rank the chromosomes, but there is no guarantee on how many chromosomes survive. This can be fixed by changing the threshold at each iteration, either to an average or to a reasonable value.
\subsection{Pair Selection}
This part of the algorithm determines how chromosomes are paired up for crossovers (creating new chromosomes). This pairing up is important as it can influence how genes spread throughout the population, and can prevent good genes being overshadowed by poor ones. All of the below are independent of the cost function.i.e they don't require it to be of a certain form, such as differentiable. 

\subsubsection{Adjacency}
Just pair adjacent chromosomes from top to bottom. If the population is sorted, this pairs chromosomes according to their ranks, which produces chromosomes without introducing much diversity (exploitation as opposed to exploration). It's very simple to implement, and every chromosome is used in crossovers. 

\subsubsection{Random}
Randomly choose pairs. To generate the pair, a chromosome is chosen randomly, twice. This gives us unmatched diversity (exploration over exploitation), which gives a higher chance to find better quality offspring. Again very simple to implement, but for a completely random selection there's a chance we choose the same chromosome twice in a single pair.

\subsubsection{Rank-Weighted Roulette}
Choose pairs with probabilities according to their ranks.  To generate the pair, two chromosomes are chosen according to the probabilities and paired together. If random can be taken as given each chromosome an equal chance to be chosen, Rank Weighted Pairing gives each chromosome a chance proportional to their rank in the population, which requires us to sort the population. This favours chromosomes with better costs. With a population of size $N_{keep}$, this probability for chromosome $C_n$ of rank n is calculated as:

\begin{equation}
    P(C_n) = \frac{N_{keep} + 1 - n}{\sum_{i=1}^{N_{keep}} i}
\end{equation}

This gives probability \(1/sum(N_{keep})\) to the last chromosome and \(N_{keep}/sum(N_{keep})\) to the first. This gives a large difference in probability between adjacent chromosomes, regardless of their actual weights. For small populations, this gives a very skewed probability.

\subsubsection{Cost-Weighted Roulette}
Choose pairs with probabilities according to their weights. This doesn't need us to sort the population, but we do need to account that some costs may be negative. This is fixed by normalising the costs:

\begin{equation}
    NCost(C_n) = Cost(C_n) - C_{N_{keep}+1} +1
\end{equation}

 $C_{N_{keep}+1}$ is the cost of the best chromosome not chosen. For $X_{rate}$ selection this is obvious, for threshold selection or where $N_{keep} = 0$, instead use 2 * the cost of the lowest chromosome. As this value is guaranteed to be $\geq$ the lowest chromosome, subtracting it guarantees that all Ncosts are $<$ 0. With the normalised costs, probability for each chromosome is calculated as:
 
\begin{equation}
    P(C_n) = \frac{NCost(C_n)}{\sum_{i=1}^{N_{keep}} NCost(C_i)}
\end{equation} 

This gives a more accurate reflection of the relationships between the weights of the chromosomes, but is more computationally expensive. 

\subsection{Tournament Selection}
A small subset of chromosomes is randomly chosen, and the best performing chromosome from this subset. This is done twice to get a pair, and the size of each subset is normally 2-3. A variation involves taking a larger subset, ranking them and then performing \emph{rank-weighted roulette} to choose two chromosomes.

Both variations favour chromosomes with better costs. Also, since in each instance only a trivial subset has to be sorted, it reduces computation on sorting and thus works well for problems with large populations. 
\subsection{Crossover}
Once the parents are selected, to generate offspring we take parts from each and combine them to make new chromosomes. Other than a few exceptions these don't generate new values: just re-arrange the existing ones. Repeatedly doing just crossovers only covers a limited section of the search space, so isn't sufficient to find an optimal solution. Crossover methods can be combined and altered: for example, a Single Point crossover followed by Blending, or only considering a subset of indices for Extrapolation.

Only the first three crossovers are applicable to the Binary Genetic\\ Algorithm. Crossovers for ordering problems are detailed in a separate section.

\subsubsection{Single Point}
This breaks each parent into two parts, and creates offspring by swapping the second part for each. A random number \(0\leq n < L\) is generated, and used to split the chromosomes. L is the length of the binary string / number of genes for binary and continuous genetic algorithms respectively. Using slice notation, this can be shown as: 

\begin{align}
    Child_1 &= Parent_1[0:n] + Parent_2[n:end] \nonumber\\
    Child_2 &= Parent_2[0:n] + Parent_1[n:end] \nonumber
\end{align}

\subsubsection{Double Point}
This breaks each parent into 3 parts, and swaps the middle parts for each. This middle section is denoted by two crossover points, which are the start and end of the section respectively. These follow \(0\leq n_1 < n_2 \leq L\). Using slice notation, this can be shown as: 

\begin{align}
    Child_1 &= Parent_1[0:n_1] + Parent_2[n_1:n_2] +  Parent_1[n_2:end] \nonumber\\
    Child_2 &= Parent_2[0:n_1] + Parent_1[n_1:n_2] + Parent_2[n_2:end] \nonumber
\end{align}

\subsubsection{Uniform Crossover}
This uses a variable $\mu \in (0,1)$ to determine how much crossover occurs. For each index of the binary string/gene tuple, generate a random number n $\in (0,1)$. If $n < \mu$, the bit/gene in that index is swapped in the children, otherwise are directly copied from the parent. The addition of $\mu$ gives us a tuning parameter, but if it's too high or low effectively no crossover will happen (no swaps or all swaps).

\subsubsection{Blending}
This uses a variable $\beta \in (0,1)$ to combine the genes from the parents (P) into new values as below:

\begin{align}
    Child_1[i] &= \beta * P_1[i] + (1-\beta) * P_2[i] \nonumber\\
    Child_2[i] &= \beta * P_2[i] + (1-\beta) * P_1[i] \nonumber  
\end{align}

These values are in the range (P\_1[i], P\_2[i]), so while these generate new values they are still limited to a subset of the search space. $\beta$ can change from generation to generation, etc. 

\subsubsection{Extrapolation}
This also uses a variable $\beta$, but there's no limit on its value. Children are generated using the below:

\begin{align}
    Diff_i &= abs(P_1[i] - P_2[i]) \nonumber \\
    Child_1[i] &= P_1[i] - \beta * Diff_i \nonumber\\
    Child_2[i] &= P_2[i] + \beta * Diff_i \nonumber  
\end{align}

With some re-arrangement these are the same equations as Blending, but as there's no limit on $\beta$ this can explore the entire search space without mutation.  $\beta$ can change from generation to generation, and if needed we can artificially limit the range of values generated. 
\subsection{Mutation}
Mutation is the process that actually drives evolution: by changing the chromosomes into new values, new solutions can be evaluated and used. It is possible to have an algorithm that consists entirely of mutation: this would perform very similar to random walk or (1+1)ES. Crossovers move the genes into potentially better solutions, mutations discover new solutions. 

\subsubsection{How many Mutations?}
Mutation is determined using a mutation rate, $\mu$. To prevent good solutions from being mutated, we can designate the top n chromosomes as \emph{elite}, and not consider them for mutation. n can be left as 0, in which case any chromosome can be mutated. 

With $N_{pop}$ as the size of the population, the standard formula is:

\begin{equation}
    M = (N_{pop} - n) * \mu * N_{bits}
\end{equation}

$N_{bits}$ is the length of the binary string for a chromosome for BGA, and the number of genes in a chromosome for CGA.

\subsubsection{Binary Mutation}
Disregarding the top n chromosomes, choose M bits randomly across the population and flip them (0$\rightarrow$1, 1$\rightarrow$0). 

\subsubsection{Numeric Mutation}
This uses a variable $\sigma$ to control how much a gene mutates. Disregarding the top n chromosomes, choose M genes randomly and apply:

\begin{equation}
    Gene = Gene + \sigma * N_n(0,1)
\end{equation}

$\sigma$ is usually a constant, but for a dynamic system can be changed between generations. $N_n$ is a random number generator, but generates numbers according to a standard distribution (i.e 0.5 is most likely to be generated, 0 and 1 are very unlikely). 
\subsection{Permutation Problems}
Certain problems (such as the \textbf{Travelling Salesman Problem}) require their solution to be an ordering of a given set of variables. This means traditional mutation and crossover methods won't work, since they change the values and discover new ones respectively. Chromosomes take the form of a tuple of values, similar to CGA, but every chromosome has the same values in a different order.



\subsubsection{Partially Matched Crossover}
Perform two-point crossover, then swap duplicated genes \emph{outside} of the two points. This can be done by going through the children, and swapping the first duplicate in Child$_1$ with the first in Child$_2$ and so on .e.g.

\begin{align}
    Parent 1: (3,4,6,2,1,5)\; &; \;  Parent 2: (4,1,5,3,2,6) \nonumber \\
    Chosen Points&: 1, 3 \nonumber \\
    Initial Children&: (3|1,5|2,1,5); (4|4,6|4,2,6) \nonumber \\
    Fixed Children&: (3,1,5,2,4,6); (1,4,6,3,2,5) \nonumber 
\end{align}

This doesn't preserve the existing ordering in the slightest: so two very good orderings could potentially produce two terrible ones. 

\subsubsection{Ordered Crossover}
Similar to two-point crossover, choose two points and add the middle section to the children. Then fill in the empty spaces with the genes from the parent starting at the first crossover point, skipping the duplicates. e.g.

\begin{align}
    Initial &: Parents - (3|46|215), (4|15|326) \nonumber \\
    Phase 1 &: Children - (x|15|xxx), (x|46|xxx) \nonumber \\
    Phase 2 &: Children - (4|15|623), (1|46|532) \nonumber 
\end{align}

For Child$_1$, the x's are filled in with 4623 (starting at crossover point 1, skipping duplicates, looping around at the end). For Child 2, 1532 is taken from the parent. An alternative is to delete the duplicates from the parents then directly fill in the children, which saves skipping the duplicates.

This method preserves the relative ordering, but not the absolute. For TSP this is fine since the start is irrelevant, and the sequence can just be rotated to get the desired start point.
\newpage
\subsubsection{Cycle Crossover}
Swap the left-most genes, then keep swapping duplicates from left to right in child$_1$ till there are none left. This method has a relatively simple implementation, as only child$_1$ is checked for duplicates and the checking index only advances from left to right (looping around if necessary).e.g.

\begin{align}
    Initial &: Parents - (|346215), (|415326) \nonumber \\ 
    Swap 1 &: Children - (4|46215), (3|15326) \nonumber \\
    Swap 2 &: Children - (41|6215), (34|5326) \nonumber \\
    Swap 3 &: Children - (41622|5), (34531|6) \nonumber \\
    Swap 4 &: Children - (4163|25), (3452|16) \nonumber 
\end{align}

\subsubsection{Coding for Crossovers}
Rather than a direct crossover method, the chromosomes are encoded to a numeric string, can then have any crossover method applied to them, then decoded back to a chromosome. \\ \\
Encoding:
\begin{enumerate}[label=\Alph*]
\item Make a reference list of the genes
\item For idx in range(0,len(chromosome)):
\begin{enumerate}[label=\arabic*]
    \item coded[idx] = index of chromosome[idx] in reference list
    \item remove chromosome[idx] from the reference list
\end{enumerate}
\end{enumerate}
e.g. 
\begin{table}[H]
\centering
\begin{tabular}{|c|c|c|c|}
\hline
Epoch & Coded & Chromosome & Reference List \\ \hline
1 &  & 346215 & 1,2,3,4,5,6 \\ \hline
2 & 3 & 46215 & 1,2,4,5,6 \\ \hline
3 & 33 & 6215 & 1,2,5,6 \\ \hline
4 & 334 & 215 & 1,2,5 \\ \hline
5 & 3342 & 15 & 1,5 \\ \hline
6 & 33421 & 5 & 5 \\ \hline
7 & 334211 &  &  \\ \hline
\end{tabular}
\end{table}
Decoding:
\begin{enumerate}[label=\Alph*]
\item Make a reference list of the genes
\item For idx in range(0,len(coded)):
\begin{enumerate}[label=\arabic*]
    \item chromosome[idx] = index of coded[idx] in reference list
    \item remove coded[idx] from the reference list
\end{enumerate}
\end{enumerate}
e.g. 
\begin{table}[H]
\centering
\begin{tabular}{|c|c|c|c|}
\hline
Epoch & Coded & Chromosome & Reference List \\ \hline
1 &  351311 & & (,2,3,4,5,6 \\ \hline
2 &  51311 & 3 & 1,2,4,5,6\\ \hline
3 &  1311 & 3,6 & 1,2,4,5\\ \hline
4 &  311 & 3,6,1 & 2,4,5\\ \hline
5 &  11 & 3,6,1,5 & 2,4\\ \hline
6 &  1 & 3,6,1,5,2 & 4\\ \hline
7 &  &  3,6,1,2,5,4 & \\ \hline
\end{tabular}
\end{table}

\subsubsection{Mutation}
As generating new values breaks the problem, the following methods can be used:

\begin{enumerate}
    \item Invert/Reverse a randomly selected subset of the chromosome
    \item Move/Swap a randomly selected subset within the chromosome
    \item Randomly swap positions: this may break an encoded chromosome, so it must be decoded first
\end{enumerate}


\subsection{Genetic Programming}
This application of GA allows it to build a program for us. It is provided an input and a desired output, and the algorithm builds us a program to convert between to two.

Rather than strings or tuples, chromosomes are represented as tress of varying sizes and shapes. This requires extra overhead for each chromosome, since the size and shape has to be tracked as well.
Lead nodes are variables and constants, while internal tree nodes are functions that combine these.

The algorithm also has to be provided a set of possible values for variables/constants, a set of functions, and semantics on how to combine the two. e.g. the SUM function requires two numerics and outputs a numeric, the OR function requires two binary strings and outputs a binary string, etc.

\subsubsection{Crossovers}
The simplest crossover is to swap and replace subtrees between chromosomes. A single child can be generated by replacing a subtree in one parent by one from the other parent, and two children can be obtained by swapping subtrees.

\subsubsection{Mutation}
\begin{description}
    \item [Node Mutation] Swap a node with an appropriate alternative\\ e.g replace SUM with DIFF, replace 4 with 11
    \item[Swap Mutation] Swap the order of terminal nodes provided to a function\\ e.g a DIFF b $\rightarrow$ b DIFF a
    \item[Grow Mutation] Replace a terminal node with a new subtree that provides the same type as the replaced node e.g. replace 5 with a SUM 7
    \item[Gaussian Mutation] Add a value determined by a gaussian ($N_n * \sigma)$ to a terminal node
    \item[Trunc Mutation] Replace a subtree with an appropriate terminal node\\e.g replace 5 MOD 2 with 11
\end{description}
\subsection{Schema Theorem}
This theorem is used to demonstrate that GAs are effective in discovering good solutions, and increase the quality of the solutions with more generations. The proof uses the binary genetic algorithm, but the result proves effectiveness for all GAs.

\subsubsection{What is a Schema}
A schema (or scheme) is a template that represents binary strings, similar to how regular expressions represent regular strings. They consist of '0','1' and '*' (wildcard).

The order of a schema is the number of fixed positions it contains (O(s)), and the defining length ($\delta$(s)) is the distance between the first and last fixed positions. e.g.

\begin{align}
    **00&**0*** \nonumber \\
    Order&: 3 \nonumber \\
    \text{Defining Length}&: 7-3 = 4  \nonumber \\
    \text{Examples of matching Strings}&: 1100010101, 0000000000, 0100100111 \nonumber
\end{align}

For this proof, we define the average cost of a schema s at  generation t (f(s,t)) as the average cost of all the chromosomes that match the schema at that generation, and n(s,t) is the number of chromosomes at t that match s.

\subsubsection{Schema Pair Selection}
Using cost-weighted roulette, the probability that a chromosome C is chosen is \(\frac{Cost(C)}{\sum_{i=1}^{N_{keep}} Cost(C_i)} \).
Therefore, for a chromosome matching a schema at the generation, the average \\probability of being chosen is  \(\frac{f(s,t)}{\sum_{i=1}^{N_{keep}} Cost(C_i)} \). The probability for \textbf{any} chromosome \\ matching the schema being chosen is \(\left(n(s,t) * \frac{f(s,t)}{\sum_{i=1}^{N_{keep}} Cost(C_i)} \right)\).\\
This selection happens $N_{pop}$ times. If we assume that only selection takes place and no crossover/mutation occurs, we obtain:
\begin{equation}
    n(s,t+1) = n(s,t) * \frac{f(s,t)}{\sum_{i=1}^{N_{keep}} Cost(C_i)} * N_{pop} = n(s,t) * \frac{f(s,t)}{\text{Avg. Cost at gen t}} 
\end{equation}
With some re-arrangement, we can re-write these as the following:
\begin{align}
    n(s,t+1) &= n(s,t) * (1 + \varepsilon(s,t)) \\
             &= n(s,t-1) * (1+\varepsilon(s,t-1)) * (1+\varepsilon(t)) \nonumber \\ 
             &= n(s,1) * \prod_{i=1}^{t}(1+\varepsilon(s,t)) \\
    \varepsilon(s,t) &= \frac{f(s,t)-Avg.cost(t)}{Avg.cost(t)} 
\end{align}

\subsubsection{Schema Crossover}
After being selected as a parent, a schema is said to survive a crossover if one of the child chromosomes also matches it. The only circumstance a schema breaks in single-point crossover is if the break point is inside the defined length of the schema: as the externals are wild cards it doesn't matter. Using this, and the knowledge that the last index can't be selected for a single-point crossover, the probability that a schema survives a single-point crossover is: 
\begin{equation}
    P(survival) = 1- \frac{\delta(s)}{len(s)-1}
\end{equation}

\subsubsection{Schema Mutation}
Schemas are resilient to mutation as wildcards match both 0 and 1, so they only break if one of the fixed positions is mutated. If $P_m$ is the probability for a bit to be mutated, the probability a schema survives mutation is:
\begin{equation}
    P(survival) = (1 - P_m)^{O(s)}
\end{equation}

\subsubsection{Conclusion}
Combining the above three sections, we obtain the schema growth equation:

\begin{align}
    P(survival) &= n(s,t) * (1 + \varepsilon(s,t)) * \left(1- \frac{\delta(s)}{len(s)-1} \right) * (1 - P_m)^{O(s)} \\
    &= n(s,t) * \frac{\text{Avg.Cost of Matches}}{\text{Overall Avg. Cost}} * (1- \frac{Defining Length}{Schema Length}) * (1- P_m) ^ {Order(s)}
\end{align}

Therefore, we can say that above-average, short, low-order schema propagate well. The more above average the better, so with enough generations eventually the weaker solutions disappear and the more optimal ones remain. 
\end{document}
