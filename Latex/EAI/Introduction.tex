\section{Introduction}
\emph{Philosophy} is the study of general and fundamental questions about life, the universe and everything. It can be split into:
\begin{description}
    \item[Metaphysics] The nature of reality - what exists, how is it ordered, what is it like? \\ \quad -  e.g. Is there a God, What gives things identity (Ship of Thesus)
    \item[Epistemology] the study of knowledge - what is knowledge, how can we know things? 
    \\ \quad -  e.g. What makes something true? (e.g. 'fake news')
    \item[Ethics] what we 'should' do and what is good to do
    \\ \quad -  e.g. What is good? Is morality objective/subjective/relative
    \item[Logic] the study of arguments and reasoning
\end{description}

\subsection{Cogito Ergo Sum}
\emph{I think therefore I am} (cogito ergo sum) is a phrase written by Rene Descartes (pronounced \emph{dec-art}) in 1637 - it means that the act of thinking implies there is a self that does the thinking.\\ However, the rest of the world could be a complete illusion (e.g. the matrix), even one's on physical body. The only certain existence is the self, and thus it must be fundamentally different from the world, including the body.\\
This is a fundamental part of western culture (e.g. the belief that your thoughts had no influence on your body) - but not in eastern culture (e.g. Indian/Buddhism believe in the effects of meditation). Western culture is slowly starting to come round though.

\subsection{Free Will and Moral Responsibility}
\emph{Free Will} is the conscious experience of making a choice. However, according to physics it's possible to draw up a complex deterministic cause-effect chain of events or processes all the way back to the big bang. Therefore, any 'choice' we make is deterministic based on processes in the brain and external factors. e.g. any criminal that commits a crime didn't freely choose to, but rather it was a series of causes and reasons \\

\emph{Moral Responsibility} is when a person/agent/self decided to do something while having the option to do something else. If free will doesn't exist, then they never made that choice, it was just a natural consequence of the process. Therefore if an agent does something wrong, they shouldn't be punished (\emph{retribution}), rather the punishment should halt or change the process to prevent it from happening again (\emph{rehabilitatement)}.\\

If we consider the sub-atomic random nature of particles, the series of events is no longer deterministic - it's completely random. However, it's still a chain of events, not free will. 


\subsection{Logic in Philosophy}\label{subsec:logic_in_philosophy}
Logic is used in philosophy to rationalise/find holes in arguments in a standard framework. It can be used to find weaknesses/contradictions by standardising the format (e.g.if X then Y, but an exception $X_1$ can be found). \\

Knowledge often grows/improves through being defeated/refutation (e.g. we learn new things about the world by learning that our prior assumptions were wrong). We update our knowledge to resolve the contradictions (e.g. if X and $X \neq X_1$ then Y).

\subsection{Reasoning in AI}
An AI system could be perfectly logical, but without knowing what contradictions/etc. to look for it isn't a useful system - the problem is giving an AI 'common sense reasoning'. Humans are perhaps unique in this sense, as the only creatures on Earth to have a  sense of consciousness, ethics, morality, etc. (though studies are showing that other animals may have limited versions of these traits) 

\subsubsection{Natural Selection vs. Design}
Humans have evolved through natural selection over millions of years - but Artificial Intelligence will be directly developed by intelligent creatures (i.e. humans), and thus will not be subject to many of the 'restrictions' that we are.\\

One example is how humans are tuned to look at the intended outcome of our actions, rather than the side effects (this could have been due to hunter/gatherers needing to focus on goals). A missile strike on a munitions factory (where civilians work) is acceptable, but not a strike directly on civilians. \\

AI won't have this restriction, as it doesn't need to evolve for survival. e.g. in AI Planning, planners look at every effect an action has rather than just the desired one

\subsection{Can an AI be considered alive?}
The definition of life varies, but a popular one is 'organisms which are open systems
that maintain homeostasis (a steady internal physical state), are composed of cells, have a life cycle, undergo
metabolism, can grow, adapt to their environment, respond to stimuli, reproduce
and evolve'. By this, AI can never be alive. \\

Another is ‘Self-replicating information-processing system whose information 
determines both its behaviour and the blueprints for its hardware’ - this perfectly describes computers - the information is software. \\

In Life 3.0 by Max Tegmark, he gives definitions on different tpes of life:

\begin{enumerate}
    \item Life 1.0 (Biological) - The hardware/software stay the same in life and slowly evolve. e.g animals, bacteria
    \item Life 2.0 (Cultural) - Hardware is constant, software is shaped during life. i.e. it learns and grows throughout life . e.g humans are shaped by society, and learn/grow throughout life
    \item Life 3.0 (Technological) Both the hardware and software can be shaped by life - beings that can adjust their bodies as they do their behaviour. e.g. AI-powered robots
\end{enumerate}
We can say that humans are in life 2.1 right now, with things like pacemakers and prosthetics. 