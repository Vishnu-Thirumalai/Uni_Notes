\section{Ant Colony Optimisation}
An type of swarm optimisation where the agents mimic ants, using sign-based stigmergy. Though the base form solves shortest path and TSP problems, a large number of problems can be reduced to these forms.\\
Problems are given in a graph structure, with each edge (i,j) having a cost/length and pheromone concentration ($\tau_{ij}(t)$) associated with it. Also defined are $L^k(t)$, which is the length of the path taken by ant k at generation t, and $N^i_k(t)$, the list of nodes reachable from node i for ant k at gen t. The latter two will become more clear in the next section. \\
The overall flow of the algorithm is: 
\begin{enumerate}[label=\Alph*]
\item Create random initial pheromone concentrations ($\tau_{ij}(0)$)
\item While stopping criteria are unmet:
\begin{enumerate}[label=\arabic*]
    \item Create and evaluate a path for each ant 
    \item Evaporate the Pheromones
    \item Apply feedback to the Pheromones 
\end{enumerate}
\item Select the path with the most ants/ the most optimal path\\
\end{enumerate}

\subsection{Creating an Ant Path}
At each generation/iteration, a path is created for each ant according to the transition probabilities. These probabilities vary for each method (and are detailed later), but the overall method for shortest path problems is:\\
\begin{enumerate}
    \item [I] $X^k(t)$ = \{start\_node\}, i = start\_node
    \item While i != goal\_node:
    \begin{enumerate}
        \item If $N^i_k(t)$ is empty, append the predecessor nodeto i
        \item Select a node j from $N^i_k(t)$ according to the transition probabilities
        \item Append (i,j) to $X^k(t)$, i \,=  \,j
    \end{enumerate}
    \item Remove all loops from $X^k(t)$
    \item Calculate $L^k(t)$
\end{enumerate}
Removing the loops from $X^k(t)$ affects $L^k(t)$, so care must be taken: for example, there may be two conflicting loops of different lengths, so depending on which one is removed this changes the attractiveness of the path. 

\subsection{Simple Ant Colony Optimisation}

\subsubsection{Transition Probability}
SACO just uses a pheromone-weighted roulette for its transition probabilities: these are evaluated at each node while creating a path. The probability that ant k chooses edge (i,j) from node i at generation t is:

\begin{equation}
    P_{ij}^k(t) = 
    \begin{cases}
         $$ \frac{\tau_{ij}(t)}{\sum_{ab \in N^i_k(t)} \tau_{ab}(t)} $$ &, if $$ j \in N^i_k(t) $$ \\
        0 &, otherwise
    \end{cases}
\end{equation}

\subsubsection{Pheremone Evaporation}
This uses a variable, $\rho \in (0,1)$, to determine how much the pheremones evaporate at each stage. The evaporation removes the effects of older pheromones on current probabilities, so for a path to be consistently picked the pheromones on it need to be reinforced.

\begin{equation}
    \tau_{ij}(t) = (1-\rho) * \tau_{ij}(t)
\end{equation}

\subsubsection{Pheremone Updates}
SACO uses a simple update system: pheromones for an edge are increased based on the number of ants that included that edge in their path (path for ant k is $X^k(t)$), and this increase is based on the quality of the ants' solutions. It uses a constant, Q, to moderate the increase in the pheremones. 

\begin{align}
    \tau_{ij}(t+1) &= \tau_{ij}(t) + \sum_{k \in ants} \Delta \tau_{ij}^k(t) \\
    \Delta \tau_{ij}^k(t) &= 
    \begin{cases}
         $$ \frac{Q}{L^k(t)}  $$ &, if $$ (i,j) \in X^k(t) $$ \\ 
        0 &, otherwise
    \end{cases}    
\end{align}

