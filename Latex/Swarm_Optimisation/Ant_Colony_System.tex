\newpage
\subsection{Ant Colony System}
ACS builds even further of AS, providing more tuning parameters and increasing the system's exploration. It also slightly changes the functions of Pheromone Evaporation and Pheromone Updates: they become global and local updates respectively, where global affects all pheromones and local only those on the best path.

\subsubsection{Transition Probability}
It uses a user-provided variable $r_0 \in (0,1)$ to balance between exploration and exploitation (higher $r_0 \rightarrow$ more exploitation). From an ant k at node i, the edge taken is decided by:

\begin{equation}
    \text{Next edge for ant k} = 
    \begin{cases}
    \text{Lowest cost edge} \in  N^i_k(t) &, \text{if rand(0,1)} < r_0 \\
    \text{Use Exponent-based from Ant System} &, \text{otherwise}
    \end{cases}
\end{equation}


\subsubsection{Pheromone Evaporation (Local Update)}
Unlike previous systems, this both evaporates and adds to the pheromone concentration. This local addition gives the system more chance to explore since all edges get a boost, but this addition remains constant so its effects is reduced in later generations when the system is closer to finding an optimal path. It uses the user-provided constants $\rho_L \in (0,1)$ and $\tau_o > 0$. If either were 0 then it would be regular pheromone evaporation, and if $P_L$ was 1 then the pheromone intensities would be reset at every generation.

\begin{equation}
    \tau_{ij}(t) = (1-\rho_L) * \tau_{ij}(t) + \rho_L * \tau_o
\end{equation}

\subsubsection{Pheromone Updates (Global Update)}
The global updates gives the edges on the global best path an extra bonus, similar to the elite pheromone updates from AS, and thus pushes the system towards exploitation. It also adds an extra layer of evaporation, which reduces the universal bonus given in the previous step. The global best path ($X^+(t)$) can either be the best in this generation ($\Tilde{x}(t)$) or the best from the first to the current generation ($\hat{x}(t)$). This update also uses a user-provided constant $\rho_G \in (0,1)$, which again isn't 0 or 1 for the reasons mentioned above. 

\begin{align}
    \tau_{ij}(t+1) &= (1-\rho_G) * \tau_{ij}(t) + \rho_G * \Delta \tau_{ij}(t) \\
    \Delta \tau_{ij}(t) &=    
    \begin{cases}
         $$ \frac{1}{L^{X^+}(t)}  $$ &, if $$ (i,j) \in X^+(t) $$ \\ 
        0 &, otherwise
    \end{cases}        
\end{align}
