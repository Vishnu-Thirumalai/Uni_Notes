\section{Table of Prefixes}
A \textbf{Prefix Table} for a string X is a table t such that:
\begin{equation}
    t[i] = | lcp( X, X[i:]) | \nonumber
\end{equation}
i.e the length of the lcp between X and suffix $X_i$. Naively, this can be constructed in $O(n^2)$, but by using previously computed values this can be $O(n)$. \\ \\

\subsection{Proofs for Algorithm}
For $i > 1$:
\begin{align}
    g &= max(j+t[j], 0< j <i) \nonumber \\ 
    f &= j \emph{(This is the j from g)} \nonumber
\end{align}
Using the above values, X[f:g-1] is a prefix of X of length t[j] (g-1-f = t[j], and we know t[j] is the length of a match). Therefore, X[f:g-1] is a border of X[0:g-1]. \\ \\
Using the above, if $\boldsymbol{i < g}$ we can say that:
\begin{equation}
    t[i] = 
    \begin{cases}
        pref[i-f] &, if pref[i-f] < g-i \\
        g-i &, if pref[i-f] > g-i \\
        g-i + |lcp(x[g-i:|x|-1], x[g:|x|-1]| &, otherwise
    \end{cases} \nonumber
\end{equation}
As a border, x[0:t[f]]=x[f:f+t[f]-1]. Since f+t[f]=g and $i<g$, i $\in$ x[0:t[f]].  
\begin{enumerate}
    \item If t[i-f] $<$ g-i, then the entire match occurs within x[0:t[f]]. The same match will occur for t[i] since it's the same string, so we just copy the value. 
    \item g-i is the distance between i and the mismatch. The same match occurs for t[i], so we know that the new string only matches till g as well, so the length of the match is g-i.
    \item The match occurs to the end of the substring in the original, but we don't know if it ends or continues to match, so we check the remainder.
\end{enumerate}
\newpage
\subsection{Prefix Table Algorithm}
\textbf{Prefixes(X)}:
\begin{enumerate}[label=\Alph*]
    \item \(pref[0]= |X|, g=0, f = undefined\)
    \item for i in range(1, n):
\begin{enumerate}[label=\arabic*]
    \item if $ i < g$ and $pref[i-f] \neq g-i $:
    \begin{enumerate}
        \item [] pref[i] = min(pref[i-f],g-i) \emph{(Case 1 and 2)}
    \end{enumerate}    
    \item else:
    \begin{enumerate}
        \item g = max(g,i),   f=i \emph{(if $i < g$ Case 3, otherwise regular comparison)}
        \item while \( g < |x| \) and \( x[g] = x[g+f]\)
        \begin{enumerate}
            \item [] g++ 
        \end{enumerate}   
        \item pref[i] = g-f
    \end{enumerate}  
\end{enumerate}    
\end{enumerate}
This algorithm runs in $O(|x|)$: ii can run at most $|x|$ times across the entire run since g only increases, and B only runs $|x|$ times, so the total is $O(2|x|-1)$. \\ \\
This algorithm can also be used for string matching: append the pattern to be found P to the string S, then find pref[i]=len(P).
