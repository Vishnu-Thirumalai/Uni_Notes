\section{Boyer-Moore Algorithm}
One of the most generally efficient string matching algorithms, variations of this are often used in text editor for find/replace. Unlike previous algorithms, it checks the pattern from \textbf{right to left}. It still looks for the pattern from the beginning of the string (moves the window \textbf{left to right}), but within the window it checks from right to left. \\
When the algorithm finds a mismatch/matches the entire string, it uses one of two shifts to move the window to the right. For both of these, we take the original/search strings as y/x, the left-most position of the window as j (i.e y[j] is compared against x[0]), and $|x|$ as m. 

\subsection{Good-Suffix Shift}
Assuming the mismatch happens at x[i] (y[i+j]), we can say that the substrings\\ x[i+1:m-1] and y[j+i+1:j+m-1] are equal. This string is stored as u, and the mismatched character from y as c. The window is shifted so that the right-most occurrence of u in x with a different preceding character lines up with y[j+i+1:j+m-1]. This occurrence is normally to the left of the currently aligned u, so the window is shifted to the right to align it with u in y. The right-most is to prevent accidentally skipping over a match. \\
If there isn't such an occurrence, instead find the longest suffix of u that matches a prefix of x and align them: x is shifted to the right to match u in y. U is a suffix of x, and we're trying to align it with a prefix of x: this is finding a border, so we shift by per(x).

\subsection{Bad-Character Shift}
Again assuming the mismatch happens at x[i] (y[i+j]), we store the mismatched character from y as c. The window is shifted to align y[i+j] with the right-most occurrence of in x[0:m-2]. Again, the right-most is to prevent accidentally skipping over a match. If there is no occurrence of c in x, j is set to j+m+1: effectively shifting the entire string by its full length.

\subsection{Pre-Processing}
Both of these shifts can be pre-processed for the search string (x), and the algorithm checks the table for both to decide which gives the larger shift. 

\subsection{Good-Suffix}
Since this uses suffixes of x, we denote each suffix of x as $U_i$. We then define d(u) = $min|z|$ such that:
\begin{enumerate}
    \item TUZ is a suffix of X, but TU isn't a suffix of x
    \item [] OR
    \item x is a suffix of uz
\end{enumerate}
Z is a string matching one of these conditions, T is an single character belonging to $\sum$. Using the above, we get d(u) = per(uz).\\ \\
Proof of the above:
\begin{enumerate}
    \item Since uz and u are both suffixes of x, and $uz > u$, u is a suffix of uz $\rightarrow$ therefore also a border of uz, and z is a period for uz. \\ Also, since we're searching left to right, and TU isn't a suffix of x, T can't be the character that caused the mismatch. Therefore this gives us the an occurrence of u to the left of the current position that has a different preceding character. Since we're looking for $min|z|$, this gives us the first occurrence.  
    \item Since u is a suffix of x and x is a suffix of uz, u is also a suffix of uz $\rightarrow$ This means u is a border of uz, so z is a period of uz. \\
    This gives us the a suffix of u that matches a prefix of x. Since we're looking for $min|z|$, this gives us the longest suffix.
\end{enumerate}
Z effectively shifts U to the left, to align it with a new occurrence or find the longest prefix. Since this new occurrence is to the left in X, we shift X to the right to line it up with Y. Since U shifts by $|z|$, and we have to shift all of X to align U in the same way, we shift $|x|-|z|$: in rare cases this might be less than 0, so we set the shift to min(0, $|x|$-$|z|$). \\

\noindent To efficiently compute this, we compute the suffix table for x (see section \ref{sec:Suffix_Table}). 
After that, we compute d(u) for each u: first we try to find the periods (Case 1), and then check to see if it's more efficient to find prefixes of x in its suffix (Case 2).\\ \\
\textbf{Compute\_D(suff)}:
\begin{enumerate}[label=\Alph*]
    \item \(|X| = m, j=0, D[i] = m\; \forall\; i \)
    \item for i in range(m-1, 0):
\begin{enumerate}[label=\arabic*]
    \item if $ \text{suff}[i] = i+1 $:  \emph{This is a border: see the bottom of \ref{sec:Suffix_Table}}
    \begin{enumerate}
        \item [] while \( j < m-i-1\) \emph{i+1 is a border, $|x|-i-1$ is a period. This loop runs through the length of this period.}
        \begin{enumerate}
            \item D[j] = m-i-1 \emph{Sets each member of the period string to the period (Case 1) (i.e the shift to get the same character)}
            \item j ++
        \end{enumerate}    
    \end{enumerate}    
    \item for i in range(0, m-2)
    \begin{enumerate}
        \item D[m-suff[i]-1] = m-i-1 \emph{m-suff[i]-1 is the appearance of that character at x[m-i-1] in the suffix of the prefix x[0:1] (Case 2) (m-i-1 is the shift to match them)}
    \end{enumerate}  
\end{enumerate}    
\end{enumerate}

\newpage
\subsubsection{Bad-Character}
Unlike every other table so far, this stores a value DA for each character in $\sum$. DA[c] for a character c in x is index of the right-most occurrence in $x[0:|x|-2]$, \textbf{with the indices reversed }.i.e the distance between the right-most occurrence in $x[0:|x|-2]$ and the end of $x[0:|x|-2]$. This guarantees that $DA[c] \geq 2$ If c doesn't appear in x, DA[c]=$|x|$.\\
To convert from DA[c] to the shift:
\begin{equation}
    Shift = DA[c]-|x|+1+i 
\end{equation}
Where i is the mismatched position in x (this is the regular index, even though the checking is done from right to left), and c is the mismatched character in y. 

\subsection{Algorithm + Performance}
With the pre-processed shifts the algorithm is relatively simple: it checks each window from right to left, then uses whichever shift gives it a larger right shift.\\
\textbf{BM(x,y)}:
\begin{enumerate}[label=\Alph*]
    \item \(j = 0  \)
    \item while \(j < |y| - |x|\):
\begin{enumerate}[label=\arabic*]
    \item i = $|x|-1$ \emph{(j/i are for y/x respectively)}
    \item while $i \geq 0$ and $x[i]=y[i+j]$: \emph{(Checking left to right)}
    \begin{enumerate}
        \item [] i++
    \end{enumerate}    
    \item if $i = -1$: 
    \begin{enumerate}
        \item print(X is present in Y at j)
        \item j+=per(x)
    \end{enumerate}      
    \item else:
    \begin{enumerate}
        \item [] j+= max(D[i], DA[y[j+i]]-m+i+1)
    \end{enumerate}  
\end{enumerate}    
\end{enumerate}
At worst, this is $O(|x||y|)$: but it's usually closer to $3|y|$, and at best can be $\frac{|y|}{|m|}$. It's better than MP/KMP for natural language processing, but for periodic sequences like DNA it's worse.

\newpage
\section{Karp-Robin Algorithm}
One weakness of most algorithms is that they have to check each character of x and y at each stage. The Karp-Robin instead hashes x, then compares it with the hash of the window in y: only if they're the same (so the strings are similar) does it compare the actual characters. \\ 
An immediate downside of this is that the window can only be shifted by 1 each time, since we can't use periods/borders/any other shortcuts on the string, However, by building a hash such that each hash can be constructed from the previous one, this algorithm can become very efficient: even O(n) on average.

\subsection{Constructing the Hash}
A hash is a function that converts from some range of values to a limited set of integers. Using a modulus is a  popular hash function, as x\%y converts any integer x to the range (0,y-1). It's not perfect though, as \(x\%y=z\%y\) does not imply $x=y$, which might provide false positives. \\
A useful property of the modulus is: 
\begin{equation}
    ((x\%q) + (y\%q)) \% q  = (x+y) \% q \nonumber
\end{equation}
Using this property, we can make a hash such that the hash of $x[i_1:i_2]$ can be used to generate the hash of $x[i_1+1:i_2+1]$. The hash for a string s, using user-parameters d and q, is:
\begin{equation}
    hash(s) = (\sum_{i=0}^{|s|-1} (int)s[i] * d^{m-1-i})\; \boldsymbol{\%}\; q \nonumber
\end{equation}
Effectively, convert each character to a number (this could be ASCII, or more likely custom-based on the language), multiply it by the appropriate power of d (so each position has a weight . i.e $hash(ab)\neq hash(ba)$), then modulus the whole thing. The weight to each position and each character having a custom value makes the hash for each string quite unique before modulus, but without the modulus the numbers could quickly become very large. \\
To generate the next hash (removing s[0] and adding c, the next character in the searched string):
\begin{equation}
    new\_hash = ((hash(s) - ((int)s[i]*d^{m-1})) * d + (int) c)\; \boldsymbol{\%}\; q \nonumber
\end{equation}
Effectively, remove the first term (which has weight $d^{m-1}$), increase each character's weight in the hash by one position, then add in the new character with weight 1 (implying the last position) and modulus the whole hash. By the above modulus properties, this is the same as just generating the new hash, but is $O(1)$ rather than $O(n)$.
\newpage 
\subsection{Algorithm + Performance}
Again, with the hash the algorithm is relatively simple: shift the window by 1 at each iteration, and if the hashes are equal check the characters.
\textbf{KR(x,y,d,q)}:
\begin{enumerate}[label=\Alph*]
    \item \(h_x=0, h_y=0, D = d^{|x|-1}  \)
    \item for i in range(0, $|x|-1$): \emph{(Generating hash(x) and initial hash(y))}
    \begin{enumerate}
        \item \(h_x = h_x * d + (int) x[i] \)
        \item \(h_y = h_y * d + (int) y[i] \)
    \end{enumerate}    
    \item for j in range(0,\(|y|-|x|\)): \emph{(Since the window starts at j, we don't need windows starting at the last $|x|$ chars)}
\begin{enumerate}[label=\arabic*]
    \item if $h_x$ = $h_y$
    \begin{enumerate}
        \item i = 0
        \item while $i < |x|$ and $x[i]=y[i+j]$: \emph{(Checking left to right)}
        \begin{enumerate}
            \item [] i++
        \end{enumerate}      
        \item if $i = |x|$: 
        \begin{enumerate}
           \item print(X is present in Y at j)
        \end{enumerate}          
    \end{enumerate}
    \item if \(j!= |y|-|x|\): \emph{(y[i+j] is out of bounds)}
    \begin{enumerate}
        \item \(h_y = (h_y - (y[j]* D) )* d + (int) y[i] \)
    \end{enumerate}  
\end{enumerate}    
\end{enumerate}
Again, at worst this is $O(|x||y|)$ if the hash is terrible (e.g. hash(s) = 1). However, if $q \geq m$ and assuming that the strings are evenly distributed in the range (0,q) by \textbf{\%q}, then the number of false positives is at worst n/q (all the strings with equal hashes are false matches). The algorithm is then $O(n + \frac{nm}{q})$, so overall $O(n)$. \\
This algorithm is also great for multiple string searching: we generate a hash for each string to be found, and they can all be compared at the same time, rather than running through and checking for each string separately. 