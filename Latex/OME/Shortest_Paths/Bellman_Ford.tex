\subsection{Bellman-Ford Algorithm}
We've seen in previous sections that for a graph (v,e), the shortest path between any two nodes/vertices has at most v-1 edges. We've also seen that relaxing an edge reduces the bound on the destination node, and this relaxation spreads out from the source node. Therefore, if we relax every edge in the graph v-1 times, we guarantee that the bound on every node is equal to the shortest path weight. \\ \\
\textbf{Bellman\_Ford(V,E)}:
\begin{enumerate}[label=\Alph*]
    \item v.d = $\infty$, v.$\pi$ = Nil $\forall$ v $\in V$
    \item s.d = 0
    \item for i in range(0, v-1): \emph{(range(a,b) takes values a to b-1)}
\begin{enumerate}[label=\arabic*]
    \item [] for (a,b) in E:
    \begin{enumerate}
        \item [] relax(a,b)
    \end{enumerate}
\end{enumerate}    
    \item for (a,b) in E:
\begin{enumerate}[label=\arabic*]
    \item [] if (a,b) can be relaxed
    \begin{enumerate}
        \item [] return False
    \end{enumerate}
\end{enumerate}  
    \item return True
\end{enumerate}
Since every shortest path is less than v-1, if an edge can still be relaxed after step C that means there's a negative cycle: this would give us a weight of $-\infty$, so the algorithm just rejects it by returning false.\\
This algorithm is \textbf{O(VE)} - it loops through all the edges V times in total (v-1 times in step C, 1 time in step D).

\subsection{Improvements}
\begin{itemize}
    \item Edges only need to be updated when their source nodes are updated: the current algorithm updates every edge every time, this can be reduced by keeping a track of which nodes get updated. 
    \item Most shortest paths aren't going to be v-1 edges long: the loop can terminate early if there are no edges to relax at a generation. A shortest path of length k will finish relaxing every node in it in generation k.
    \item The negative cycle check can be done in the first loop: by keeping a track of the updated nodes, if there are any nodes to relax at iteration v then there's a negative cycle
\end{itemize}


