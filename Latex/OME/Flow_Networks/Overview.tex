\section{Flow Networks}
Flow networks are directed graphs with two additions:
\begin{itemize}
    \item Each edge (u,v) has a capacity c(u,v)
    \item Two distinct nodes are marked as the source and sink 
\end{itemize}
Usually, optimising a flow network is finding the maximum flow: the amount of units that can travel from the source to the sink simultaneously. Units travel along the edges, but an edge can only support up to its capacity simultaneously. Other problems may be the minimum flow that hits a certain target, or having multiple types of flow at the same time. \\
Most of the below algorithms assume that if an edge (a,b) exists, then the edge (b,a) also exists. If it's not present in the given graph, it's given a default capacity of zero.

\subsection{Flow}
The flow f(u,v) of an edge (u,v) is the amount of units passing through the edge: it's abstracted to a function that maps edges to real numbers. Flow is subject to the following:
\begin{enumerate}
    \item f(a,b) $\leq$ c(a,b) $\forall$ (a,b) - Flow for an edge is always less than/equal to capacity
    \item For all nodes except source/sink, incoming flow = outgoing flow (net flow on a node = 0)
    \item If f(a,b) $>$ 0, then f(b,a) = 0
\end{enumerate}
A \textbf{saturated edge} is one where f(a,b) = c(a,b). The \textbf{value of a flow} in a network is defined as the net flow from the source/net flow into the sink.\\
A flow can be decomposed into \textbf{paths or cycles}, which start from the source and end at the sink. There can be at most m paths/cycles, where m is the number of edges.

\subsection{Flow-Feasibility Problems}
A variation on flow networks, rather than a single source/sink each vertex has a value d(v). Usually \(\sum_{v \in V}d(v) = 0\).
\begin{itemize}
    \item If $d(v) > 0$, it's a \textbf{Supply Node}. Supply nodes have an outgoing flow (negative).
    \item If $d(v) < 0$, it's a \textbf{Demand Node}, with a demand of abs(d(v)). Demand nodes have an incoming flow (positive).
    \item If $d(v) = 0$, it's a \textbf{Transitional Node}.
\end{itemize}
These problems try to find a flow such that d(v) = 0 for all nodes, and can be reduced to max. flow problems.

\subsubsection{Reduction to Max. Flow Problems}
\begin{enumerate}
    \item Make a source node and connect it to each supply node s with an edge of capacity d(s).
    \item Make a sink node and connect it to each demand node x with an edge of capacity d(x).
    \item If the max. flow saturates the edges from the source/into the sink, that's a feasible flow.
\end{enumerate}




