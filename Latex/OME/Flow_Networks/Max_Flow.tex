\subsection{Max Flow}
More of a general approach than an algorithm, this uses the concept of residual capacity to iteratively build the optimal flow. The following algorithms use the ideas here. 

\subsection{Residual Capacity}
The residual capacity for an edge (a,b) is defined as:
\begin{equation}
    c_r(a,b) = 
    \begin{cases}
        c(a,b) - f(a,b) &, if f(a,b) > 0 \\
        c(a,b) + f(b,a) &, if f(b,a) > 0
    \end{cases}
\end{equation}
By the definition of a flow, if $f(u,v)>0$, then f(v,u)=0. The residual capacity of an edge with a flow is how much more flow it can take (capacity-flow), and the backward edge has how much flow can be reversed (capacity + reverse edge flow). \\
A \textbf{Residual Edge} is an edge with residual capacity $>0$. The \textbf{Residual Network} of a flow f is the flow network made of residual edges after applying f. This network can be used to find flows not included in f, and these can be 'added' to f.

\subsection{'Adding' Flows}
By using the residual network, there might be a flow which improves the current one: this flow is 'added' to the current flow, altering it and increasing the overall flow. Assuming there are two flows $F_1$ and $F_2$, adding them together is done edge-by-edge to produce the flow $F'$. Assuming that $F_1(u,v) > 0$ (and therefore $f(u,v)=0$):
\begin{itemize}
    \item If $F_2(u,v) > 0$, then $F'(u,v) = F_1(u,v) + F_2(u,v)$
        \begin{itemize}
            \item Both flows are in the same direction for this edge, just add them together
        \end{itemize}
    \item If $F_2(v,u) > 0$, then $F'(u,v) = max(0, F_1(u,v) - F_2(v,u))$ and $F'(v,u) = max(0, F_2(v,u) - F_1(u,v))$
        \begin{itemize}
            \item The flows run in opposite direction, so think of them as colliding
            \item The larger flow direction is set to the difference of the flows
            \item The smaller flow direction is set to 0
        \end{itemize}    
\end{itemize}

\subsection{Algorithm}
\begin{enumerate}[label=\Alph*]
    \item F = Zero Flow \emph{(f(a,b) = 0 for all (a,b))}
    \item while True: 
\begin{enumerate}[label=\arabic*]
    \item N = Residual Network of F
    \item F' = Flow in N
    \item If F' does not exist:
    \item [] \quad break
    \item F = F.add(F')
\end{enumerate}  
\item return F
\end{enumerate}