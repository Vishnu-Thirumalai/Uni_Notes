\subsection{Cuts in a Flow Network}
A \textbf{Cut} is splitting the nodes in a network into two groups: group S contains the source, group T contains the sink. 
\begin{itemize}
    \item The \textbf{Capacity} of a cut is $\sum c(u,v)$ where u $\in$ S, v $\in$ T, (u,v) $\in$ E : the capacity of the edges going from S to T
    \item The \textbf{Net Flow} of a cut is $\sum f(a,b) - \sum f(c,d)$ where a,c $\in$ S, b,d $\in$ T, (a,b),c,d $\in$ E : the flow from S to T minus the flow from T to s
\end{itemize}
For any cut, flow(cut) = flow(sink) $\leq$ capacity(cut) (equal when there is no flow from S to T). Therefore, \textbf{max. flow in a network $\boldsymbol{\leq}$ min. capacity of all cuts}.

\subsubsection{Minimum Cut}
A minimum cut of a graph is the cut with the lowest capacity: there may be more than one. One method to find minimum cuts:
\begin{enumerate}
    \item Find the maximum flow, and generate its residual network N
    \item Make a list a = \{S\}, where s is the source 
    \item Add all nodes reachable from s, then all nodes reachable from the new nodes, etc. until no more nodes are reachable
    \begin{itemize}
        \item If t (the sink) is reached, then this isn't a maximum flow because there's a path from s to t
    \end{itemize}
    \item The first minimum cut is (\{a\}, \{v-a\}) (v is the list of vertices/nodes)
    \item Repeat the following steps till t is reached:
    \begin{enumerate}
        \item Add a single node that connects to a
        \item Continue to add all new reachable nodes (as in step 3) until no more nodes are reachable.
        \begin{itemize}
            \item If at any point t is added to the list, stop and end the algorithm.
        \end{itemize}
        \item The next minimum cut is (\{a\},\{v-a\})
    \end{enumerate}
\end{enumerate}

\newpage
\subsubsection{Max-Flow Min-Cut Theorem}
\begin{enumerate}
    \item Max. flow in a network $\leq$ min. capacity of any cut in the network
    \item The following three statements are equivalent:
    \begin{enumerate}
        \item F is a max. flow in G 
        \item There's no augmenting path in the residual network $G_F$
        \item For some cut (S,T) in G, c(S,T) = f(S,T) = F 
    \end{enumerate}
\end{enumerate}

Proof:
\begin{description}
    \item [a$\rightarrow$b] If there was an augmenting path, it could be used to increase the flow. The flow can't be increased, so there's no augmenting path
    \item [b$\rightarrow$c] No augmenting path means all the paths from source to sink are saturated, which means there are no reverse edges. All edges from S to T in a cut will be at capacity, all edges from T to S will be empty.
    \item [c$\rightarrow$a] When c(S,T) = F(S,T), we've shown that this is a max flow. Since f(S,T) = F, F is a max flow
\end{description}

\noindent This is also a proof for the Ford\_Fulkerson algorithm: \textbf{when there are no augmenting paths left, a max. flow has been found}. 
