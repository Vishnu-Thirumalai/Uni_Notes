\section{What is Optimisation?}

\subsection{Combinatorial Optimisation}\label{sec:Combinatorial_Optimisation}
Most problems can be represented as (R,C) where:
\begin{description}
    \item[R] is a \textbf{finite} set of configurations, each of which is a possible solution to the problem. They effectively represent the search space of the problem, and in most cases won't be explicit.e.g.\\
    S={1,2,3,\dots ,100} and S = ${x|x\in N, x\leq 100}$ are the same space, but the second is much more compact
    \item[C] is the cost function, which tells us the value of each configuration. It takes in a configuration and outputs a real number. 
\end{description}
\textbf{An optimisation problem is to find the configuration with the lowest cost}\\(i.e.$\underset{R}{\operatorname{arg\; min}}(r)$).If we wanted to find the configuration with the highest cost, we could just multiply the costs by -1.

\subsubsection{SAT Problem}
Given a boolean CNF (a boolean consisting of 'AND' with 'OR' as sub-problems\\(e.g. (a OR b) AND (c or d))), find a configuration of truth values for each variable that satisfies the CNF (i.e. the whole CNF results in true). \\
The cost function is simply 1 if the CNF is true, and 0 otherwise. X-SAT refers to a problem where the longest sub-problem's length is X. Any X $>2$ can be converted to a 3-SAT by adding extra variables.  2-SAT is P complexity, 3-SAT is NP-Complete (Explained in \ref{subsec:Time_Complexity_Problems}).



\subsection{Time Complexity}

\subsubsection{Algorithms}
Measuring the running time of algorithms is futile: it depends on the hardware used, the background processes running on the computer, etc. Instead it's assumed that each step takes a constant amount of time, and the growth in the number of steps based on the size of the input determines the time complexity of an algorithm. \\ 
Again, rather than comparing the exact number of steps, an algorithm is abstracted to the largest influence on the number of steps. e.g. If the algorithm takes \(4n^2+2n+3\) steps (n is the input size), we represent only consider the $n^2$. This is because, for large enough input values, the lower powers have little impact on the final value. e.g. The difference between $n^2$ and $n^2+n$ at n = 1000 is 0.1\%. 
\newpage
To represent the time complexity of an algorithm (f(x), the growth in the number of steps with the growth in the input size), we use the following three notations:
\begin{description}
    \item[Big-O(O)] An algorithm is O(g(x)) if there are some constants c and d such that \(c*g(x) \geq f(x)\) when $x > d$. This gives an upper bound on the performance of an algorithm, so the \textbf{Worst Case}, and is the notation mostly used.
    \item[Big-Omega($\Omega$)] An algorithm is $\Omega$(g(x)) if there are some constants c and d such that \(c*g(x) \leq f(x)\) when $x > d$. This gives an lower bound on the performance, so the \textbf{Best Case}.
    \item[Big-Theta($\Theta$)] An algorithm is $\Theta$(g(x)) if there are some constants $c_1$, $c_2$ and d such that \(c_1*g(x) \leq f(x) \leq c_2*g(x)\) when $x > d$. This gives a tight bound on the performance, so the \textbf{Average Case}.    
\end{description}
O and $\Omega$ could be very loose but still technically correct. e.g f(x) = n  is O($n^{10}$), but that doesn't tell us anything useful. Since we mostly expect large input values, we use d to remove problems at smaller values: e.g. \(5n > n^2\) for \(n < 5\), but $n^2$ grows quicker,

\begin{center}
   \( O(log n) < O(n) (Linear) < O(n * log n) < O(n^x) (Polynomial) < O(X^n) (Exponential) \) \\
    \(O(n!) \approx O(n^5). \)
\end{center}

\subsubsection{Problems}\label{subsec:Time_Complexity_Problems}
\begin{description}
    \item[P] denotes the class of problems for which there exists a \textbf{Polynomial-Time} algorithm to solve them. These are computationally 'easy'. e.g. Shortest Path with Positive Weights, Linear Programming
    \item[NP] problems can't be solved, but a solution can be verified, in polynomial time. Alternatively, they can be solved in polynomial time by a \textbf{Non-deterministic Turing Machine} (a type of automaton with memory). Finding a polynomial solution to one of these is an ongoing problem. e.g. Travelling Salesman Problem, Boolean SAT
\end{description}
Chess is NP-Hard (as hard/harder than every other NP problem), but is EXP-Complete (an exponential-time algorithm), since verification is exponential. A problem is NP-Complete if it is harder than/as hard as every other NP problem, and is also an NP problem itself.

\subsection{Travelling Salesman Problem}
The input is a graph (V,E) with the cost of each edge $\in E$ known. The objective is to find the path that visits every vertex $\in V$ exactly once with a minimal cost. Some problems may require the path to start and end at the same node.\\ \\
This can be converted to a decision problem: given a length X, is there a path with cost $< X$? Alternatively, each vertex can be given as an (x,y) point in space, and the euclidean distance (\(\sqrt[2]{(x_1-x_2)^2+(y_1-y_2)^2}\)) between points is used as the cost. The classic problem is NP-Hard, the decision problem is NP-Complete.

