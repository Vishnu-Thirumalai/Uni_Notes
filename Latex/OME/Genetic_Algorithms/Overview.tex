\section{Genetic Algorithms}
Genetic Algorithms work by evolving a pool of candidate solutions (a population-based method), optimising them in the process till it eventually converges to an optimal enough solution. This is a \textbf{meta-heuristic}: its a way to find a way to find an optimal solution. \\
Each configuration is represented as an \textbf{individual}, which contains values for the variables of the given problem. \textbf{Binary Genetic Algorithms} represent individuals as a string of binary characters of length n. BGAs assume that \textbf{all} binary strings of length n represent a valid configuration: if this isn't true then there's an overhead in checking validity.\\
One change for this version of the Genetic Algorithm is to define the \textbf{fitness function} of a configuration: how good a solution is. This version tries to find the individual with the highest fitness function: this can be obtained by multiplying the cost function by -1.
\\ \\
This form of the genetic algorithm has 4 main parts:
\begin{enumerate}
    \item Natural Selection
    \item Pair Selection
    \item Crossover
    \item Mutation\\ 
\end{enumerate}
Combining all of these together, the overall flow of the algorithm is: 
\begin{enumerate}[label=\Alph*]
\item Create a random initial population
\item While stopping criteria are unmet:
\begin{enumerate}[label=\arabic*]
    \item Create pairs via Pair Selection
    \item Make new individuals via Crossover
    \item Filter population with Natural Selection
    \item Apply mutation to the population
\end{enumerate}
\item Select the best performing individual as the solution\\
\end{enumerate}
For problems that have constraints or multiple objectives to be optimised, having the cost function be a weighted sum of each individual requirement is sufficient. 

\subsection{Stopping Criteria}
Any/all of the below can be used: it depends on the situation:
\begin{enumerate}
    \item An acceptable (high enough fitness) solution has been found
    \item The mean/standard deviation/range of the individuals is within a target value
    \item No change in individuals best/cost/worst cost after X iterations
    \item N iterations have elapsed
\end{enumerate}