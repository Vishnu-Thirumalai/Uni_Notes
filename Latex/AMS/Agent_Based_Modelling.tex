\section{Agent Based Modelling}
\textbf{Simulations} are an abstracted model of the real world. They can be used to test theories/effects of potential future changes, without actually implementing them. \\

They're widely used where data analytics fail.i.e in very complex systems, where behaviour is emergent, and where data isn't available. Simulations \emph{generate} data, rather than just analyse it.\\

\subsection{Models}
Any simulation uses a model, which is an abstraction of the real world made to be as simple as possible. Models are based on assumptions and theories, often to test the theories. \\

If a simulation/model explains a real-world phenomenon, then it is a \emph{candidate solution} - further proof is needed to call it a \emph{causal explanation}. The further proof is an ill-posed problem.\\

One test is to vary the initial parameters - fragile models are very sensitive, so will vary dramatically with a tiny change. 

\subsection{Bottom-Up Modelling}
Agent-based modelling is a \emph{bottom up} approach - the individual parts of the system are defined, and the behaviour of the system as a whole is emergent from the interactions of the parts. \\

These parts/agents can be heterogeneous - a single agent can then be individually changed, and the effects of that change are then transmitted to the system organically.

\subsection{Environment}
Agents are embedded into an environment, which define their behaviour.
\begin{description}
    \item [Spatial Models] Agents have co-ordinates on a grid, and can move about. Interactions are based on proximity.
    \item [Network Models] Links between agents are via edges, like a graph. Interactions are based on link proximity.
\end{description}
Environments can also contain non-autonomous entities, like the weather/ external factors.

\subsection{Scale and Feedback}
Since agents are non-deterministic, and behaviour emerges from interactions, every run of the system is unique. These can be from lots of simple agents, or a few behaviourally complex agents. e.g. Testing network effects at scale on simple agents vs. testing how a few agents react to a complex environment

The rich data obtained comes from feedback loops: the final state isn't a simple function of the initial state. These loops don't depend on the size/scale of the system, just its structure. 

\subsection{Cellular Automata}
A simpler system that influenced most ABM models. \begin{itemize}
    \item The environment is a grid of cells, and each cell has a state
    \item At each time step, the cell updates based on its neighbours' states
    \item Each cell is a non-moving, homogeneous agent with no memory in a fixed dynamic environment (i.e the neighbouring cells)
\end{itemize}
A famous example is Conway's Game of Life.

\subsection{Shelling's Segregation Model}
A precursor to ABM's, that focused on individuals in the system - their individual decisions had effects they neither intend nor were aware of. e.g people wanting to live amongst their community splits cities on racial lines

\subsection{Intervention}
An \emph{Intervention} is a change in the model that affects the agents behaviour. It can be explored by comparing the 'before' and 'after' applying the changes.

\subsubsection{Influence Maximisation}
By identifying which agents have most influence on the system (e.g the most connections in a network model), interventions can be targeted at them - this maximises the amount of change with minimal intervention. In the real world, these could be applying changes to social media influencers or targeted at certain ethnic/political groups.

\subsubsection{Influence/Spreading Functions}
Influence can be modelled in different ways, such as in peer pressure (an agent is influenced if a certain percentage of its connections are). The spread of disease can be modelled in such a fashion. 

\subsection{Creation}
Usually the agents are relatively simple (e.g. BDI, state machines, subsumption-based), and the behaviour emerges from their interactions. The agents only need to capture the relevant behaviour, so they don't need to be complex. \\

Data is required for realistic modelling:

\begin{itemize}
    \item Calibration - to set the initial agent/environment parameters
    \item Verification - use the data to set limits on the agent behaviour, etc.
    \item Validation - check the simulation is correct against real data (Face validation is checking against estimates)
\end{itemize}

\subsubsection{Pyschological Agents}
For social setting models, we simulate a particular aspect(s) of the psychology of the individuals. e.g wanting to do X\\

Adding psychological models such as stress and trust can increase the realism of the model - some models even have their agent learn and adapt as the model iterates (i.e. reinforcement learning)

