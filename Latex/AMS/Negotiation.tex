\section{Negotiation}
Self-interested agents need to agree on how to perform/allocate tasks/resources - they do this by negotiating.

An \emph{Agreement Protocol/Mechanism} defines the structure of the negotiation/argument:
\begin{enumerate}
    \item What moves each agent can make
    \item What are the permitted responses/sequence of moves. e.g. can an agent make an offer in response to an offer?
    \item How the outcome is determined
\end{enumerate}
Each agents has a \emph{Strategy} to decide what move it makes in a given state. 

\subsection{Game Theory}
A way to analyse negotiations/scenarios where each agent tries to maximise its own utility. 
\begin{description}
    \item[Strategy] The action the agent chooses to take
    \item[Outcome] The result of each agent/player executing their action
    \item[Dominant Strategy] No matter what the other player(s) do, this is the best strategy for an agent
    \item[Nash Equilibrium] An outcome where no agent benefits from changing, assuming no other agent changes
    \item[Pareto Optimal] An outcome where changing to any other outcome to benefit one player would harm another player(s)
    \item[Max. Social Welfare] The outcome with the max. utility sum across all agents
\end{description}
e.g.
\begin{table}[H]
\tiny
\centering
\resizebox{\textwidth}{!}{%
\begin{tabular}{|c|c|l|l|}
\hline
 Agents &  & \multicolumn{2}{c|}{Agent 2} \\ \hline
 & Actions & \multicolumn{1}{c|}{A} & \multicolumn{1}{c|}{B} \\ \hline
\multirow{2}{*}{\begin{tabular}[c]{@{}c@{}}Agent \\ \\ 1\end{tabular}} & A & \begin{tabular}[c]{@{}l@{}}\quad \quad2\\ 2\end{tabular} & \begin{tabular}[c]{@{}l@{}}\quad \quad4\\ 0\end{tabular} \\ \cline{2-4} 
 & B & \begin{tabular}[c]{@{}l@{}}\quad \quad0\\ 4\end{tabular} & \begin{tabular}[c]{@{}l@{}}\quad \quad0\\ 0\end{tabular} \\ \hline
\end{tabular}%
}
\end{table}
In the above example of a game with two agents and two actions each: BB is the Nash Equilibrium; AA, AB and BA are Pareto Optimal; AA, AB and BA maximise Social Welfare. 

\subsection{Properties of Agreement Mechanisms}
An agreement mechanism can be any of the following:
\begin{enumerate}
    \item Individual Rational - No agent is worse off for participating
    \item Symmetric - Same rules for everyone
    \item Fair - Every agent has an equal chance of affecting the outcome
    \item Pareto Optimal - A PO outcome is guaranteed
    \item Maximises social utility - An outcome that maximises social utility is guaranteed
    \item Stable - A Nash Equilibrium Exists
    \item Simple - easy to find the best strategy
    \item Guaranteed to terminate - no way the participants interact indefinitely 
\end{enumerate}

\subsection{Negotiation Definitions}
Negotiations usually take place over a series of rounds, and each agent makes one proposal per round. The \emph{Negotiation Set} is the set of possible proposals that an agent can make.

Negotiations can be \emph{single-issue} (e.g. the price of X) or \emph{multi-issue} (e.g a contract with multiple clauses). In the single it's obvious where concessions are made, but in the multi it's not obvious how the various issues relate. Multi-issue negotations also can't be simply iterated over, as there are too many possibilities. 

Negotiations can be one-to-one, many-to-one (e.g an auction) or many-to-many.

\subsection{Alternating Offers Protocol}
A one-to-one protocol to split a given amount - in each round, one agent proposes a deal and the other either accepts (and the negotiation terminates) or rejects and continues to a new round with the roles reversed.

The game is not guaranteed to terminate: if the agents can't reach a deal, this is known as the \emph{Conflict Deal} - which is the worst possible outcome for both agents. 

\subsubsection{Ultimatum Game}
Like the AoP, but only a single round. Guaranteed to terminate, but unfair to agent 2 as only agent 1 can affect the outcome (both want to avoid a conflict deal)

\subsubsection{Strategies}
Given that all agents want to avoid a conflict deal, the agent going last (with a fixed number of rounds) can offer any deal, and the other agent must accept. For an indefinite number of rounds, the first agent will get any deal they want - as they always go first, they can propose the same deal every time and reject whatever the second agent proposes. \\

Agents are generally impatient - the sooner they get something, the more value it has. This can be represented as the value of X to agent i at round K being $\delta^k_i \cdot X$ - the higher $\delta_i$, the more patient agent i is.\\
Knowing the patience of the other agent allows you to make an offer they must accept - the longer the game goes on the less the item is worth to them. For an indefinite number of rounds, the deal agent 1 should make (Assuming the split is (agent 1, agent 2)) is:
\[
    \left( \frac{1-\delta_1}{1-\delta_1\delta_2},\frac{\delta_2(1-\delta_1)}{1-\delta_1\delta_2} \right)
\]

\subsection{Task Oriented Domains}
Agents have to achieve a set of tasks: they can negotiate to swap tasks to be more efficient. The domain is represented as: \textbf{$<T,Ag,c>$} = $<$set of tasks, set of agents, cost of each subset of tasks$>$. 
\begin{itemize}
    \item Encounter - Initial allocation of tasks (a set $<T_1,T_2,...,T_i>$ where $T_i$ is the allocation for agent i)
    \item Deal - The final allocation of tasks(a set $<D_1,D_2,...,D_i>$ where $D_i$ is the allocation for agent i)
    \item Cost of deal for agent i - $c(D_i)$
    \item Utility of deal for agent i - $c(T_i) - c(D_i)$
    \item Conflict Deal - the initial deal
    \item Individual Rational Deal - a deal where no agent prefers the conflict deal
\end{itemize}
The negotiation set is all deals that are individual rational + pareto optimal (i.e utility $>= 0$ for all agents)

\subsubsection{Deception}
Agents can lie about the number of tasks they have to get a better deal. 
\begin{enumerate}
    \item Hidden Task - An agent can hide a task from its encounter. This lowers the encounter cost, so reduces the individual rational strategies to give it a better deal
    \item Phantom Task - An agent can pretend to have been allocated a new task. This task can make another task (which the agent wants) look worse, so no other agent will bid for it. 
\end{enumerate}

\subsection{Monotonic Concession Protocol}
One-to-one protocol: agents simultaneously propose deals in each round. If neither agent agrees to its opponent's deal, the round continues - in the next round, each agent's proposed deal \emph{cannot be less preferable to it's opponent}. i.e it must propose a deal that is equal/better for the other agent. If neither agent concedes on their deal, the conflict deal is chosen.

\subsubsection{Zeuthen Strategy}
\begin{enumerate}
    \item The first deal an agent proposes is its most preferred
    \item The agent least willing to risk conflict should conceded
    \item An agent should only concede enough to change the balance of risk
\end{enumerate}
This strategy maintains a Nash Equilibrium, so agents can announce this strategy without losing anything. \\

The closer an agent's deal is to the conflict deal, the more willing to risk (less to lose). 
\begin{align*}
    \text{A's willingness to risk} &= \frac{utility(Deal_a) - utility(Deal_b)}{utility(deal_a)} \nonumber \\
    &= \frac{\text{Utility lost by conceding}}{\text{Utility lost by conflict}} \nonumber
\end{align*}
If $utility(deal_a)$ = 0 (i.e. the conflict deal), then the risk = 1.