\section{Argumentation}
An \emph{argument} is a reason to believe/do/want something.
\begin{itemize}
    \item Can be exchanged between agents to resolve differences and structure interactions
    \item Agents can use them to structure their internal reasoning
    \item There can be multiple arguments for/against the same thing
\end{itemize}

An argument is structured as a set of \emph{premises} to support a \emph{claim}. i.e X because Y,F and M

\subsection{Definition}
\emph{Argumentation} is the process of constructing/exchanging arguments to interact with other agents.
\begin{enumerate}
    \item Abstract Argumentation - given a set of arguments that contradict each other, find a consistent set of knowledge
    \item Argument-based Dialogue - give a standard structure for interaction in negotiation, persuasion, etc.
\end{enumerate}

\subsubsection{Terms}
\begin{description}
    \item[Database] a set of formulae/knowledge, might not be internally consistent
    \item[Argument] (premises,claim) - the premises are present in the database, and the claim follows from the premises (using classical logic). The premises are the minimal required set to prove the claim.
    \item[Attack] when two arguments conflict with each other
    \begin{description}
        \item[Rebut] $(p_1,a_1)$ and $(p_2,a_2)$, where $a_1 = \neg a_2$ 
        \item[Undercut] $(p_1,a_1)$ and $(p_2,a_2)$, where $a_1 = \neg p$ for some $p\in p_2$ 
    \end{description}
\end{description}

\subsection{Abstract Argumentation}
The internal structure of the arguments are ignored, and the focus is on which arguments attack each other - this is represented as an attack graph (A,R) (= (arguments, attacks)).\\

An argument is \emph{In} if all attacking arguments are \emph{Out}, and \emph{Out} if any argument that attacks it is \emph{In}. Arguments can also be unlabelled. e.g. if $A\rightarrow{B}$ and $B\rightarrow{A}$, without making an assumption/external arguments then neither argument can be labelled.\\

An argument that isn't attacked by anything is \emph{In} by default. 

\subsubsection{Semantics}
    \begin{enumerate}
        \item Grounded Semantics - strictly follow the rules to minimise node labelling
        \item Preferred Semantics - Use assumptions to maximise node labelling
        \begin{enumerate}
            \item Preferred Credulous - an argument is marked In under any of the preferred labellings
            \item Preferred Sceptical - an argument is marked In under all of the preferred labellings
        \end{enumerate}
    \end{enumerate}

Preferred semantics are preferred as they provide us with more information to use.

\subsubsection{Subjective Defeat Relations}
For $A\rightarrow{B}$ and $B\rightarrow{A}$, A only \emph{defeats} B if A is subjectively preferred to B - introduces preferences for certain arguments.\\
These preferences can be explicitly shown in the graph, and can attack attack relations between arguments (e.g. in the above example, a preference for A would attack $B\rightarrow{A}$). Preferences can also attack each other or, or attacks on preferences, or \dots 

\subsubsection{Defeasible Knowledge}
Knowledge can be revised, attacked, or rendered void. If counter-evidence is provided, something assumed to be true can be changed to false. The above only applies to facts - opinions can't be objectively defeated. 

\subsection{Non-Monotonic Reasoning}
Agents can change their conclusions as they learn about the world - they can shrink or grow their set of conclusions (can only grow in monotonic). \\

Classical logic doesn't work well to represent this, but argumentation naturally allows for new knowledge to attack previous knowledge. It also provides simple/intuitive ways for humans to add in knowledge, and clearly shows why agents have taken certain decisions.

\subsection{Argument Dialogues}
Different agents can engage in dialogues with each other, to share their reasons and try to change the other agents' views. Arguments dialogues can be split into: initial situation/cause, , goal of the argument, and personal aims of the agents.

\subsubsection{Dialogue types}
\begin{table}[H]
\centering
\resizebox{\textwidth}{!}{
\begin{tabular}{|l|l|l|l|}
\hline
Type & Initial Situation & Main Goal & Personal Aims \\ \hline
Persuasion & Conflicting points of view & Resolution of views & Change the other agent's views to their own \\ \hline
Inquiry & Ignorance of multiple agents & Growth of knowledge & Find/destroy a proof/incorrect belief \\ \hline
Deliberation & Need for action & Reach a decision & Influence the outcome towards the 'best' \\ \hline
Neogtiation & \begin{tabular}[c]{@{}l@{}}Conflict of interests, \\ but co-operation required\end{tabular} & Make a deal & Get the best deal \\ \hline
Info-seeking & Ignorance of a single agent & \begin{tabular}[c]{@{}l@{}}Spread knowledge \\ (general/about other agents)\end{tabular} & Gain/Hide/Show personal knowledge \\ \hline
\end{tabular}
}
\end{table}

\subsubsection{Dialogue System}
The system is defined by:
\begin{enumerate}
    \item The moves each agent can make. e.g. assert, accept, challenge, question
    \item Protocol - rules on what moves can/can't be made. e.g. you can't attack an argument you made previously, you must accept arguments you can't attack
    \item Strategy for each agent - the 'best' strategy is still a research question
    \item When the argument terminates and determining the outcome. e.g. last argument wins
\end{enumerate}